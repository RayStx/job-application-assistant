\section{教育背景}

\datedsubsection{
\textbf{Bikini Bottom University (比奇堡大学)}, 海洋烹饪与餐饮管理,\textit{博士}}
{2017.09 - 2021.06}
\textbf{论文课题}: \textbf{蟹黄堡风味持续优化与顾客情感反应研究} (Continuous Flavor Optimization and Customer Emotional Response of Krabby Patties)

比奇堡大学“海洋美食创新奖”获得者;担任学生烹饪协会会长,组织年度“海底风味节”,吸引来访顾客超过20,000人次。

\datedsubsection{
\textbf{Bikini Bottom Culinary Institute (比奇堡烹饪学院)},
水下食品科学与海洋生物营养学,
荣誉毕业生 (Summa Cum Laude),
\textit{硕士}}
{2015.09 - 2017.06}
在\textbf{海带料理创新大赛}中荣获金奖;研发的“无气泡炸海带圈”被蟹堡王采纳并推广至全市门店,销售额增长15\%。

\datedsubsection{
\textbf{Coral Reef High School (珊瑚礁高中)},
海洋生物学基础与实用餐饮服务,
GPA: 4.8/5.0,
\textit{高中}}
{2012.09 - 2015.06}

获“年度最具乐观精神奖”;校乐队小鼓手;海底科学展一等奖(发明“自动泡泡吹机”)。

\section{实习与工作经历}

\datedsubsection{\textbf{Krusty Krab (蟹堡王)} | 厨房核心团队, \textbf{主厨}, 比奇堡}{1999.05 - 至今}
\begin{itemize}
  \item 每日制作\textbf{蟹黄堡500+个},在保持100\%顾客满意度的同时,连续20+年保持零投诉记录
  \item 主导\textbf{蟹黄堡制作流程标准化}项目:引入多手协作烹饪法,将单份制作时间\textbf{缩短20\%}(50秒→40秒),在午餐高峰期将顾客平均等待时间减少\textbf{35\%}
  \item 制定并执行\textbf{配方安全保密机制},成功阻止痞老板40+次偷配方企图
  \item 研发“节日限定蟹黄堡”(如海底圣诞款、泡泡日款),在节假日实现销量\textbf{翻倍增长}
  \item 培训与指导新入职员工10+人,建立厨房培训手册,提高团队整体效率\textbf{25\%}
\end{itemize}

\datedsubsection{\textbf{Goo Lagoon (胶水湖)} | 海滩安全巡逻员, \textbf{兼职}, 比奇堡}{2010.06 - 2015.08}
\begin{itemize}
  \item 每年夏季高峰期巡逻300+小时,协助救援海底游客20+次
  \item 创办\textbf{沙雕蟹堡互动区},提升游客参与度\textbf{50\%},成为夏季热门打卡点
  \item 与派大星合作发明“泡泡救援装置”,将遇险游客转移时间缩短\textbf{60\%}
\end{itemize}

\datedsubsection{\textbf{Bikini Bottom Jellyfish Research Center (比奇堡水母研究中心)} | 研究助理, 比奇堡}{2008.09 - 2010.05}
\begin{itemize}
  \item 采集并分析水母种群迁徙数据,建立比奇堡首个\textbf{水母活动预测模型}(准确率85\%)
  \item 研究成果用于设计“水母友好捕捉技术”,获得\textbf{水母保护协会技术创新奖}
  \item 撰写并发表论文2篇(发表于\textit{Marine Creature Studies Journal})
\end{itemize}

\section{主要研究与创新项目}

\begin{enumerate}
  \item \textbf{蟹黄堡风味与情绪的关系研究 (2019-2020)}:通过顾客反馈调查(N=500),发现风味与顾客幸福指数呈显著正相关($r=0.78$),据此调整烹饪火候与调味比例,满意度提升至98\%
  \item \textbf{节日主题餐品创新 (2018-2019)}:研发“水母果冻蟹黄堡”“深海海草饮品”,在节假日销售额增加\textbf{200\%}
  \item \textbf{厨房高效协作系统 (2016)}:引入自动翻面锅铲和海水温控系统,将生产线产能提升\textbf{30\%}
  \item \textbf{泡泡造型顾客互动技术 (2015)}:为等待顾客提供“泡泡造型体验”,平均等待满意度提升至4.9/5
\end{enumerate}

\section{荣誉奖项}
\begin{itemize}
  \item 比奇堡“年度最佳员工”奖(连续27次,1999-2025)
  \item 海底烹饪大赛冠军(2022)
  \item 水母保护协会技术创新奖(2010)
  \item 海洋社区服务贡献奖(2015、2020)
\end{itemize}

\section{技能与特长}
\begin{itemize}
  \item \textbf{烹饪}:蟹黄堡制作(大师级)、海带料理创新、气泡烘焙技术
  \item \textbf{客户服务}:记住常客喜好,提供个性化菜品推荐
  \item \textbf{设备研发}:自动翻面锅铲、泡泡救援装置
  \item \textbf{跨物种协作}:与海星、章鱼、松鼠等物种建立高效合作关系
\end{itemize}

\section{兴趣爱好}
\begin{itemize}
  \item 水母捕捉、吹泡泡、空手道(师从珊迪)
  \item 研究深海新食材并尝试融合菜
  \item 组织社区派对并免费提供蟹黄堡
\end{itemize}


% 背调 - 来自ChatGPT
% 出生 & 职业起点
% 剧集设定:海绵宝宝出生于 1986 年 7 月 14 日(官方确认过),1999 年进入蟹堡王工作(与动画第一季同步)。
% 简历里写的「1999.05 - 至今」在蟹堡王工作是可行的。
% 教育背景
% 原设定中海绵宝宝从「划船学校」毕业(但多次挂科),没有明说大学经历。但为了让简历更像真实学术型求职,加入了 2008-2021 年的学历是可行的,只需假设他工作+学习并行。
% 我建议这样排时间:
% 2012.09 - 2015.06:高中(合理)
% 2015.09 - 2017.06:硕士(设定上可以当作速成课程,因为海底教育体系不同)
% 2017.09 - 2021.06:博士(合理,且可解释为蟹堡王在职进修)
% 兼职/实习经历
% Goo Lagoon 2010-2015 兼职和水母研究中心 2008-2010 不冲突,因为水母研究可以是高中前后进行的课外科研。
% 唯一要注意的是:博士和蟹堡王主厨工作重叠——但这是可行的,可以解释为「在职博士」。
% 连续最佳员工 27 次
% 1999-2025 是 27 年整,符合逻辑(每年一次)。